\section*{Introduzione}

L'inquinamento ambientale dovuto ai rifiuti plastici è un problema che influenza tutti noi, nella fattispecie per quanto riguarda la salvaguardia 
degli oceani. Le due principali fonti di inquinamento di plastica negli oceani deriva dalla pesca e dai rifiuti gettati nei fiumi. 
Per poter contrastare la seconda causa è necessario poter effettuare operazioni di monitoraggio per prevenire che rifiuti dannosi 
possano raggiungere il mare mentre per la prima occorrono strumenti che rendano il recupero dei rifiuti più semplice e meno dispendioso. 
Anche solo riconoscere, identificare e individuare i rifiuti darebbe una grossa mano allo scopo. Per questo motivo può aver senso adoperare soluzioni 
quali modelli convoluzionali per effettuare detection degli oggetti in questione per poi agire di conseguenza con la raccolta e 
la rimozione dall'acqua della plastica.