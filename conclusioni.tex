\section{Conclusioni}

Nella tabella \ref{table:general-results} abbiamo riassunto tutte le metriche dei 
modelli realizzati. Tra tutti i modelli, \texttt{medium-200-0} è risultato quello
con precisione, richiamo e mAP più alti. 

I modelli che seguivano il metodo indicato dall'articolo hanno avuto le 
prestazioni peggiori mentre il nostro tentativo con la regolazione dei iperparametri
non ha portato ai risultati sperati.

In generale comunque le prestazioni tra i vari tentativi sono abbastanza simili, non
ci sono grosse discrepanze se non qualche tendenza di un modello a essere migliore di 
un altro.

Possiamo supporre che l'iperparametro \texttt{dropout} ha aiutato a migliorare le 
performance e che per poter aumentare l'efficienza del modello sarebbe necessario
mettere mano al dataset per contrastare meglio la distribuzione sbilanciata delle
classi nelle immagini. I metodi di augmentation della libreria ultralytics non sono
così sufficienti a contrastare questa problematica nonostante sia stato d'aiuto.

In ultimo, i modelli ottenuti possono essere utili per il riconoscimento dei
rifiuti, soprattutto per quanto riguarda le bottiglie di plastica, ma devono 
essere perfezionati per ridurre la percentuale di errore e di mancata individuazione
degli oggetti interessati.


\begin{table}[htbp]
    \centering
    \begin{tabularx}{\textwidth}{lYYYc}
        \toprule
        Model & Precision & Recall & mAP50 & mAP50-95 \\
        \midrule
        \texttt{small-1203} & 0.452 & 0.450 & 0.374 & 0.172 \\
        \texttt{medium-200-0} & 0.456 & 0.459 & 0.391 & 0.186 \\
        \texttt{small-tune-03} & 0.450 & 0.451 & 0.382 & 0.170\\
        \texttt{small-tune-04} & 0.457 & 0.449 & 0.382 & 0.172 \\
        \midrule
        \texttt{nano-ukra-0} & 0.330 & 0.379 & 0.308 & 0.149 \\
        \texttt{small-ukra-0} & 0.432 & 0.406 & 0.349 & 0.178 \\
        \texttt{medium-ukra-0} &  0.405 & 0.394 & 0.324 & 0.168 \\
        \midrule
        \texttt{nano-tune-2.0} & 0.372 & 0.428 & 0.320 & 0.159 \\
        \texttt{small-tune-2.0} & 0.400 & 0.389 & 0.345 & 0.172 \\
        \texttt{medium-tune-2.0} & 0.363 & 0.487 & 0.335 & 0.166 \\
        \bottomrule
    \end{tabularx}
    \caption{Confronto validazione su test set tra i modelli testati}
    \label{table:general-results}
\end{table}